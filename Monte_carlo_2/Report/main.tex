\documentclass[a4paper,12pt,twoside]{article}
\usepackage[polish]{babel}

\addto\captionspolish{\renewcommand{\figurename}{Rys.}}
\usepackage[utf8]{inputenc}
\usepackage[english]{babel}
\usepackage[margin=2cm]{geometry}
\usepackage[table,xcdraw]{xcolor}
\usepackage{graphicx}
\usepackage{indentfirst}
\usepackage{multicol}
\usepackage{listings}
\usepackage[T1]{fontenc}
\usepackage{bigfoot} % to allow verbatim in footnote
\usepackage[numbered,framed]{matlab-prettifier}
\usepackage{filecontents}
\usepackage{blindtext}
\usepackage{graphics}
\usepackage{adjustbox}
\usepackage{float}
\usepackage{subfigure}
\usepackage{multirow}
\usepackage{colortbl}
\usepackage{anyfontsize}
\usepackage{t1enc}
\usepackage{enumitem}
\usepackage{hhline}
\usepackage{fancyhdr}
\usepackage{marginnote}
\usepackage{amsmath}
\usepackage{amsthm}
\usepackage{mathtools}
\usepackage{dirtytalk}
\usepackage{cite}
\usepackage[font=footnotesize, labelfont=bf]{caption}
\usepackage{textgreek} %for units like micro in text
\usepackage[hidelinks]{hyperref}
\usepackage{adjustbox}
\urlstyle{same}
\usepackage{physics}
\usepackage{chemfig}
\renewcommand{\figurename}{Rys.}
\title{\textbf{Projekt 2: Generatory liczb pseudolosowych o zadanym rozkładzie w jednym
 wymiarze}}
\author{Kacper Połuszejko, 412183}
\date{}
\begin{document}
\maketitle

\section*{Wstęp}

Celem ćwiczenia było skonstruowanie jednowymiarowego generatora liczb pseudolosowych o funkcji gęstości prawdopodobieństwa:

\begin{equation}
    f(x) = \frac{4}{5}(1+x-x^3), \quad x \in [0,1],
\end{equation}

oraz dystrybuancie:

\begin{equation}
  F(x) = \frac{4}{5}(x + \frac{x^2}{2} - \frac{x^4}{4}),   \quad x \in [0,1].
\end{equation}

 Do generowania liczb pseudolosowych wykorzystano schematy dla: a) rozkładu złożonego, b) łańcucha Markowa, c) metody eliminacji.

 \section{Metodyka}

\subsection{Rozkład złożony}

Dystrybuantę zapisujemy w postaci rozkładu złożonego o ogólnym wzorze:
    
\begin{equation}
    F(x) = \sum_{i=1}^n g_i H_i(x), \quad g_i \in \mathbb{R}, \quad H_i(x) : \mathbb{R} \rightarrow \mathbb{R},
\end{equation}

zatem w naszym przypadku:

\[
F(x) = \frac{4}{5}x + \frac{1}{5}\left(2x^2 - x^4\right),
\]

skąd odczytujemy:

\[
g_1 = \frac{4}{5}, \quad H_1 = x,
\]

\[
g_2 = \frac{1}{5}, \quad H_2 = 2x^2 - x^4.
\]
Funkcje odwrotne do $H1$ oraz $H2$:

\begin{equation}
    x = U, \quad x = \sqrt{1 - \sqrt{1 - U}}, \quad U \sim U(0, 1).
\end{equation}

\newpage
W związku z tym, powyższy rozkład złożony wygenerowany został wg.następującego algorytmu:

\[
U_1, U_2 \sim U(0, 1)
\]

\[
X = 
\begin{cases}

\hspace{8mm} U_2, & \text{gdy } U_1 \leq g_1 \\ 
\frac{U_2}{\sqrt{1 - \sqrt{1 - U_2}}}, & \text{gdy } U_1 > g_1 
\end{cases}

\]

\subsection{Łańcuch Markowa}

W metodzie tej generujemy ciąg  
\[
\{X_0, X_1, X_2, \ldots\}
\]
gdzie związek pomiędzy ostatnim elementem \( X_i \) a kolejnym \( X_{i+1} \) określamy na podstawie prawdopodobieństwa przejścia, które spełnia warunek \textbf{detailed balance}  
\[
T(X_{i+1}|X_i) = T(X_i|X_{i+1}) = \frac{1}{2\Delta}, \quad \Delta \in [0, 1]
\]
i prawdopodobieństwa akceptacji nowego stanu (liczby)  
\[
p_{acc} = \min \left\{ \frac{T(X_i|X_{i+1}) f(X_{i+1})}{T(X_{i+1}|X_i) f(X_i)}, 1 \right\},
\]
gdzie $T(X_i|X_{i+1})/T(X_{i+1}|X_i)$ jest równe $1$ dzięki spełnieniu warunku detailed balance. \\

Algorytm Metropolisa generowania nowego elementu w łańcuchu:
\[
U_1, U_2 \sim U(0, 1)
\]
\[
X_{i+1} = 
\begin{cases} 
x_{new} = X_i + (2U_1 - 1)\Delta, & \text{gdy } x_{new} \in [0, 1] \land U_2 \leq p_{acc} \\ 
X_i, & \text{w przeciwnym przypadku}
\end{cases}
\]


\subsection{Metoda eliminacji}

W tej metodzie wykorzystywana jest funkcja gęstości prawdopodobieństwa f(x), dodatkowo ograniczona od góry funkcją g(x), dla której dysponujemy generatorem \textbf{G}. Algorytm generowania ciągu liczb pseudolosowych przedstawia się w następujący sposób:

\[
U_1 \sim U(0, 1)
\]
\[
G_2 \sim \mathcal{G}, \quad \text{np.} \ \mathcal{G} = 1.15 \cdot U(0, 1)
\]
\[
\begin{cases} 
G_2 \leq f(U_1) \implies X = U_1 \\ 
G_2 > f(U_1) \implies \text{losujemy nową parę } U_1, G_2 
\end{cases}
\]

\subsection{Test $\chi^2$}
Dla każdej metody wykonano test $\chi^2$ i porównano uzyskane wyniki z wartością graniczną rozkładu, przyjmując poziom istotności równy $\alpha = 0.05$. Wartość statystyki testowej dla $k-1$ stopni swobody obliczono ze wzoru:

\begin{equation}
    \sum_{i=1}^k\frac{(n_i - p_iN)^2}{p_iN},
    \label{eq:5}
\end{equation}
gdzie $p_i$ to prawdopodobieństwo, że zmienna losowa znajdzie się w i-tym przedziale, $n_i$ to ilość liczb pseudolosowych w i-tym przedziale, a $N$ to całkowita liczba wylosowanych liczb.

Na podstawie porównania dla każdej metody stwierdzono, czy hipotezę $H_0$ tego, że uzyskany ciąg liczb pseudolosowych ma rozkład $F(x)$ należy odrzucić, czy nie.

\section{Wyniki}
Wygenerowano $N = 10^6$ liczb pseudolosowych o dystrybuancie $F(x)$ dla każdej z metod. \\
Dla każdego ciągu liczb sporządzono histogram o $k=10$ podprzedziałów i porównano go z funkcją $f(x)$. Następnie wykonano testy $\chi^2$. 



 \begin{figure}[h!]
    \begin{minipage}{0.55\textwidth}
        \centering
        \includegraphics[scale = 0.5]{Zlozony — kopia.jpg}
    \end{minipage}
    %\hspace{15mm}
    \begin{minipage}{0.55\textwidth}
        \includegraphics[scale = 0.5]{Metoda_eliminacji — kopia.jpg}
    \end{minipage}
    \caption{Histogramy o liczbie podprzedziałów $k = 10$ dla ciągów liczb pseudolosowych wygenerowanych za pomocą rozkładu złożonego (po lewej) oraz metody eliminacji (po prawej).  }
\end{figure}

 \begin{figure}[h!]
    \begin{minipage}{0.55\textwidth}
        \centering
        \includegraphics[scale = 0.5]{Markow.jpg}
    \end{minipage}
    %\hspace{15mm}
    \begin{minipage}{0.55\textwidth}
        \includegraphics[scale = 0.5]{Markow_0_05 — kopia.jpg}
    \end{minipage}
    \caption{Histogramy o liczbie podprzedziałów $k = 10$ dla ciągów liczb pseudolosowych wygenerowanych za pomocą łańcucha Markowa z parametrem $\Delta = 0.5$ (po lewej) oraz $\Delta = 0.05$ (po prawej).  }
\end{figure}

\newpage

Dla każdego z wygenerowanych ciągów obliczono wartość statystyki testowej $\chi^2$ zgodnie ze wzorem (\ref{eq:5}). \\

Sprawdzimy teraz, czy hipotezę $H_0$ tego, że uzyskany ciąg liczb pseudolosowych ma rozkład $f(x)$ należy odrzucić, czy nie. \\

\textbf{Rozkład złożony}:\\
$3,325 (\alpha = 0,05) < \chi^2_{k-1} = 12,52 < 19,023 (\alpha = 0.95)$ - nie odrzucamy hipotezy $H_0$. \\


\textbf{Metoda eliminacji}:\\
 $ \chi^2_{k-1} = 2,949 < 3,325 (\alpha = 0,05)  $ - odrzucamy hipotezę $H_0$. Generowany rozkład jest zbyt bliski rozkładowi zadanemu.\\

\textbf{Łańcuch Markowa ($\Delta = 0.5$)}:\\
$19,023 (\alpha = 0.95) < \chi^2_{k-1} = 23,306$ - odrzucamy hipotezę $H_0$. Generowany rozkład za bardzo odbiega od rozkładu zadanego.\\

\textbf{Łańcuch Markowa ($\Delta = 0.05$)}:\\
$19,023 (\alpha = 0.95) < \chi^2_{k-1} = 1188,33$ - odrzucamy hipotezę $H_0$. Generowany rozkład za bardzo odbiega od rozkładu zadanego. \\

Widzimy zatem, że jedyną metodą, która przeszła test, jest metoda rozkładu złożonego. Metoda eliminacji okazała się zbyt dokładna. Z kolei ciągi liczb wygenerowane za pomocą łańcucha Markowa za bardzo odbiegały od zadanego rozkładu $f(x)$. Warto jednak zwrócić uwagę, że generator ten znacznie lepiej poradził sobie dla parametru $\Delta = 0.5$ niż $\Delta = 0.05$ co widać również na (\textbf{Rys.2}). Być może przy odpowiednim doborze parametru $\Delta$ ciąg wygenerowany za pomocą metody łańcucha Markowa przeszedłby test $\chi^2$.
 
 
\end{document}
